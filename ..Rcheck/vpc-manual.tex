\nonstopmode{}
\documentclass[letterpaper]{book}
\usepackage[times,inconsolata,hyper]{Rd}
\usepackage{makeidx}
\usepackage[utf8,latin1]{inputenc}
% \usepackage{graphicx} % @USE GRAPHICX@
\makeindex{}
\begin{document}
\chapter*{}
\begin{center}
{\textbf{\huge Package `vpc'}}
\par\bigskip{\large \today}
\end{center}
\begin{description}
\raggedright{}
\item[Title]\AsIs{Create visual predictive checks in R}
\item[Version]\AsIs{0.9.1}
\item[Date]\AsIs{2017-07-25}
\item[Author]\AsIs{Ron Keizer }\email{ronkeizer@gmail.com}\AsIs{}
\item[Maintainer]\AsIs{Ron Keizer }\email{ronkeizer@gmail.com}\AsIs{}
\item[Description]\AsIs{Visual predictive checks are a commonly used diagnostic
plot in pharmacometrics, showing how percentiles for observated data compare
to those same percentiles of data simulated from a model. This R package allows
creation of VPCs without the use of PsN and Xpose4.}
\item[Depends]\AsIs{R (>= 3.1.0)}
\item[Imports]\AsIs{classInt, dplyr, reshape2, MASS, survival, ggplot2, readr}
\item[License]\AsIs{MIT + file LICENSE}
\item[LazyData]\AsIs{true}
\item[URL]\AsIs{}\url{https://github.com/ronkeizer/vpc}\AsIs{}
\item[Suggests]\AsIs{knitr, testit}
\item[VignetteBuilder]\AsIs{knitr}
\item[RoxygenNote]\AsIs{6.0.1}
\end{description}
\Rdcontents{\R{} topics documented:}
\inputencoding{utf8}
\HeaderA{vpc-package}{VPC package}{vpc.Rdash.package}
%
\begin{Description}\relax
Create Visual Predictive Checks in R
\end{Description}
%
\begin{Author}\relax
Ron Keizer \email{ronkeizer@gmail.com}
\end{Author}
\inputencoding{utf8}
\HeaderA{add\_noise}{Add noise / residual error to data}{add.Rul.noise}
%
\begin{Description}\relax
Add noise / residual error to data
\end{Description}
%
\begin{Usage}
\begin{verbatim}
add_noise(x, ruv = list(proportional = 0, additive = 0, exponential = 0))
\end{verbatim}
\end{Usage}
%
\begin{Arguments}
\begin{ldescription}
\item[\code{x}] data

\item[\code{ruv}] list describing the magnitude of errors. List arguments: "proportional", "additive", "exponential".
\end{ldescription}
\end{Arguments}
\inputencoding{utf8}
\HeaderA{add\_sim\_index\_number}{Add sim index number}{add.Rul.sim.Rul.index.Rul.number}
%
\begin{Description}\relax
Add simulation index number to simulation when not present
\end{Description}
%
\begin{Usage}
\begin{verbatim}
add_sim_index_number(sim, id = "id")
\end{verbatim}
\end{Usage}
%
\begin{Arguments}
\begin{ldescription}
\item[\code{sim}] a data.frame containing the simulation data

\item[\code{id}] character specifying the column name in the data.frame
\end{ldescription}
\end{Arguments}
\inputencoding{utf8}
\HeaderA{auto\_bin}{Calculate appropriate bin separators for vpc}{auto.Rul.bin}
%
\begin{Description}\relax
Calculate appropriate bin separators for vpc
\end{Description}
%
\begin{Usage}
\begin{verbatim}
auto_bin(dat, type = "kmeans", n_bins = 8, verbose = FALSE)
\end{verbatim}
\end{Usage}
%
\begin{Arguments}
\begin{ldescription}
\item[\code{dat}] data frame

\item[\code{type}] auto-binning type: "density", "time", or "data"

\item[\code{n\_bins}] number of bins to use. For "density" the function might not return a solution with the exact number of bins.

\item[\code{verbose}] show warnings and other messages (TRUE or FALSE)
\end{ldescription}
\end{Arguments}
%
\begin{Details}\relax
This function calculates bin separators (e.g. for use in a vpc) based on nadirs in the density functions for the indenpendent variable
\end{Details}
%
\begin{Value}
A vector of bin separators
\end{Value}
%
\begin{SeeAlso}\relax
\code{\LinkA{vpc}{vpc}}
\end{SeeAlso}
\inputencoding{utf8}
\HeaderA{bin\_data}{Function to bin data based on a vector of bin separators, e.g. for use in VPC}{bin.Rul.data}
%
\begin{Description}\relax
Function to bin data based on a vector of bin separators, e.g. for use in VPC
\end{Description}
%
\begin{Usage}
\begin{verbatim}
bin_data(x, bins = c(0, 3, 5, 7), idv = "time")
\end{verbatim}
\end{Usage}
%
\begin{Arguments}
\begin{ldescription}
\item[\code{x}] data

\item[\code{bins}] numeric vector specifying bin separators

\item[\code{idv}] variable in the data specifies the independent variable (e.g. "time")
\end{ldescription}
\end{Arguments}
\inputencoding{utf8}
\HeaderA{create\_vpc\_theme}{Create new vpc theme}{create.Rul.vpc.Rul.theme}
%
\begin{Description}\relax
Create new vpc theme
\end{Description}
%
\begin{Usage}
\begin{verbatim}
create_vpc_theme(...)
\end{verbatim}
\end{Usage}
%
\begin{Arguments}
\begin{ldescription}
\item[\code{...}] pass arguments to `new\_vpc\_theme`
\end{ldescription}
\end{Arguments}
\inputencoding{utf8}
\HeaderA{draw\_params\_mvr}{Draw parameters from multivariate distribution}{draw.Rul.params.Rul.mvr}
%
\begin{Description}\relax
Draw parameters from multivariate distribution
\end{Description}
%
\begin{Usage}
\begin{verbatim}
draw_params_mvr(ids, n_sim, theta, omega_mat, par_names = NULL)
\end{verbatim}
\end{Usage}
%
\begin{Arguments}
\begin{ldescription}
\item[\code{ids}] vector of ids

\item[\code{n\_sim}] number of simulations

\item[\code{theta}] theta vector

\item[\code{omega\_mat}] omega matrix

\item[\code{par\_names}] parameter names vector
\end{ldescription}
\end{Arguments}
\inputencoding{utf8}
\HeaderA{new\_vpc\_theme}{Create a customized VPC theme}{new.Rul.vpc.Rul.theme}
%
\begin{Description}\relax
Create a customized VPC theme
\end{Description}
%
\begin{Usage}
\begin{verbatim}
new_vpc_theme(update = NULL)
\end{verbatim}
\end{Usage}
%
\begin{Arguments}
\begin{ldescription}
\item[\code{update}] list containing the plot elements to be updated. Run `new\_vpc\_theme()` with no arguments to show an overview of available plot elements.
\end{ldescription}
\end{Arguments}
%
\begin{Details}\relax
This function creates a theme that customizes how the VPC looks, i.e. colors, fills, transparencies, linetypes an sizes, etc. The following arguments can be specified in the input list:
\begin{itemize}

\item obs\_color: color for observationss points
\item obs\_size: size for observation points
\item obs\_median\_color: color for median observation line
\item obs\_median\_linetype: linetype for median observation line
\item obs\_median\_size: size for median observation line
\item obs\_ci\_color: color for observation CI lines
\item obs\_ci\_linetype: linetype for observation CI lines
\item obs\_ci\_size: size for observations CI lines
\item sim\_pi\_fill: fill color for simulated prediction interval areas
\item sim\_pi\_alpha: transparency for simulated prediction interval areas
\item sim\_pi\_color: color for simulated prediction interval lines
\item sim\_pi\_linetype: linetype for simulated prediction interval lines
\item sim\_pi\_size: size for simulated prediction interval lines
\item sim\_median\_fill: fill color for simulated median area
\item sim\_median\_alpha: transparency for simulated median area
\item sim\_median\_color: color for simulated median line
\item sim\_median\_linetype: linetype for simulated median line
\item sim\_median\_size: size for simulated median line
\item bin\_separators\_color: color for bin separator lines, NA for don't plot
\item bin\_separators\_location: where to plot bin separators ("t" for top, "b" for bottom)

\end{itemize}

\end{Details}
%
\begin{Value}
A list with vpc theme specifiers
\end{Value}
\inputencoding{utf8}
\HeaderA{pk\_oral\_1cmt}{Simulate PK data from a 1-compartment oral model}{pk.Rul.oral.Rul.1cmt}
%
\begin{Description}\relax
Simulate PK data from a 1-compartment oral model
\end{Description}
%
\begin{Usage}
\begin{verbatim}
pk_oral_1cmt(t, tau = 24, dose = 120, ka = 1, ke = 1, cl = 10,
  ruv = NULL)
\end{verbatim}
\end{Usage}
%
\begin{Arguments}
\begin{ldescription}
\item[\code{t}] Time after dose

\item[\code{tau}] Dosing interval

\item[\code{dose}] Dose

\item[\code{ka}] Absorption rate

\item[\code{ke}] Elimination rate

\item[\code{cl}] Clearance

\item[\code{ruv}] Residual variability
\end{ldescription}
\end{Arguments}
%
\begin{Value}
A vector of predicted values, with or without added residual variability
\end{Value}
\inputencoding{utf8}
\HeaderA{plot\_vpc}{VPC plotting function}{plot.Rul.vpc}
%
\begin{Description}\relax
VPC plotting function
\end{Description}
%
\begin{Usage}
\begin{verbatim}
plot_vpc(db, show = NULL, vpc_theme = NULL, smooth = TRUE,
  log_x = FALSE, log_y = FALSE, title = NULL, xlab = "Time",
  ylab = "Dependent value", verbose = FALSE)
\end{verbatim}
\end{Usage}
%
\begin{Arguments}
\begin{ldescription}
\item[\code{db}] object created using the `vpc` function

\item[\code{show}] what to show in VPC (obs\_dv, obs\_ci, pi, pi\_as\_area, pi\_ci, obs\_median, sim\_median, sim\_median\_ci)

\item[\code{vpc\_theme}] theme to be used in VPC. Expects list of class vpc\_theme created with function vpc\_theme()

\item[\code{smooth}] "smooth" the VPC (connect bin midpoints) or show bins as rectangular boxes. Default is TRUE.

\item[\code{log\_x}] Boolean indicting whether x-axis should be shown as logarithmic. Default is FALSE.

\item[\code{log\_y}] Boolean indicting whether y-axis should be shown as logarithmic. Default is FALSE.

\item[\code{title}] title

\item[\code{xlab}] ylab as numeric vector of size 2

\item[\code{ylab}] ylab as numeric vector of size 2

\item[\code{verbose}] verbosity (T/F)
\end{ldescription}
\end{Arguments}
\inputencoding{utf8}
\HeaderA{read\_table\_nm}{NONMEM output table import function}{read.Rul.table.Rul.nm}
%
\begin{Description}\relax
Quickly import NONMEM output tables into R. 
Function taken from `modelviz` package by Benjamin Guiastrennec.
When both \code{skip} and \code{header} are \code{NULL},
\code{read\_nmtab} will automatically detect the optimal
settings to import the tables. When more than one files are
provided for a same NONMEM run, they will be combined into
a single \code{data.frame}.
\end{Description}
%
\begin{Usage}
\begin{verbatim}
read_table_nm(file = NULL, skip = NULL, header = NULL,
  rm_duplicates = FALSE, nonmem_tab = TRUE)
\end{verbatim}
\end{Usage}
%
\begin{Arguments}
\begin{ldescription}
\item[\code{file}] full file name

\item[\code{skip}] number of lines to skip before reading data

\item[\code{header}] logical value indicating whether the file contains the names
of the variables as its first line

\item[\code{rm\_duplicates}] logical value indicating whether duplicated columns should be removed

\item[\code{nonmem\_tab}] logical value indicading to the function whether the file is a
table or a nonmem additional output file.
\end{ldescription}
\end{Arguments}
%
\begin{Value}
A \code{data.frame}
\end{Value}
%
\begin{Examples}
\begin{ExampleCode}
## Not run: 
data <- read_table_nm(file = '../models/pk/sdtab101')

## End(Not run)
\end{ExampleCode}
\end{Examples}
\inputencoding{utf8}
\HeaderA{replace\_list\_elements}{Replace list elements by name}{replace.Rul.list.Rul.elements}
%
\begin{Description}\relax
Replace list elements by name
\end{Description}
%
\begin{Usage}
\begin{verbatim}
replace_list_elements(list, replacement)
\end{verbatim}
\end{Usage}
%
\begin{Arguments}
\begin{ldescription}
\item[\code{list}] original list

\item[\code{replacement}] replacement list
\end{ldescription}
\end{Arguments}
%
\begin{Details}\relax
Finds and replaces list elements by name and throws an error if an 
element is not available in the original list. This is a local duplicate
of the PKPDmisc copy for the VPC package to reduce dependency on PKPDmisc
at this time.
\end{Details}
%
\begin{Examples}
\begin{ExampleCode}
## Not run: 
list <- list(ipred = "ipred", dv = "dv", idv = "idv", "pred" = "pred")
replacement <- list(dv = "conc", idv = "time")
list <- replace_list_elements(list, replacement)

## End(Not run)
\end{ExampleCode}
\end{Examples}
\inputencoding{utf8}
\HeaderA{rtte\_obs\_nm}{Simulated RTTE data (1x)}{rtte.Rul.obs.Rul.nm}
\keyword{datasets}{rtte\_obs\_nm}
%
\begin{Description}\relax
An example dataset with simulated repeated time-to-event data
\end{Description}
%
\begin{Usage}
\begin{verbatim}
rtte_obs_nm
\end{verbatim}
\end{Usage}
%
\begin{Format}
An object of class \code{data.frame} with 573 rows and 6 columns.
\end{Format}
\inputencoding{utf8}
\HeaderA{rtte\_sim\_nm}{Simulated RTTE data (100x)}{rtte.Rul.sim.Rul.nm}
\keyword{datasets}{rtte\_sim\_nm}
%
\begin{Description}\relax
An example dataset with simulated repeated time-to-event data (100 simulations)
\end{Description}
%
\begin{Usage}
\begin{verbatim}
rtte_sim_nm
\end{verbatim}
\end{Usage}
%
\begin{Format}
An object of class \code{data.frame} with 2000000 rows and 7 columns.
\end{Format}
\inputencoding{utf8}
\HeaderA{show\_default}{Defaults for show argument}{show.Rul.default}
\keyword{datasets}{show\_default}
%
\begin{Description}\relax
Defaults for show argument
\end{Description}
%
\begin{Usage}
\begin{verbatim}
show_default
\end{verbatim}
\end{Usage}
%
\begin{Format}
An object of class \code{list} of length 11.
\end{Format}
\inputencoding{utf8}
\HeaderA{simple\_data}{A small rich dataset}{simple.Rul.data}
\keyword{datasets}{simple\_data}
%
\begin{Description}\relax
A small rich dataset
\end{Description}
%
\begin{Usage}
\begin{verbatim}
simple_data
\end{verbatim}
\end{Usage}
%
\begin{Format}
An object of class \code{list} of length 2.
\end{Format}
%
\begin{Details}\relax
a list containing the obs and sim data for an example dataset to run a 
simple vpc.
\end{Details}
%
\begin{Examples}
\begin{ExampleCode}
## Not run: 
vpc(simple_data$sim, simple_data$obs)

## End(Not run)
\end{ExampleCode}
\end{Examples}
\inputencoding{utf8}
\HeaderA{sim\_data}{Simulate data based on a model and parameter distributions}{sim.Rul.data}
%
\begin{Description}\relax
Simulate data based on a model and parameter distributions
\end{Description}
%
\begin{Usage}
\begin{verbatim}
sim_data(design = cbind(id = c(1, 1, 1), idv = c(0, 1, 2)),
  model = function(x) {     return(x$alpha + x$beta) }, theta, omega_mat,
  par_names, par_values = NULL, draw_iiv = "mvrnorm",
  error = list(proportional = 0, additive = 0, exponential = 0), n = 100)
\end{verbatim}
\end{Usage}
%
\begin{Arguments}
\begin{ldescription}
\item[\code{design}] a design dataset. See example

\item[\code{model}] A function with the first argument the simulation design, i.e. a dataset with the columns ... The second argument to this function is a dataset with parameters for every individual. This can be supplied by the user, or generated by this sim\_data if theta and omega\_mat are supplied.

\item[\code{theta}] vector of fixed effect parameters

\item[\code{omega\_mat}] vector of between subject random effects, specified as lower triangle

\item[\code{par\_names}] A character vector linking the parameters in the model to the variables in the dataset. See example.

\item[\code{par\_values}] parameter values

\item[\code{draw\_iiv}] draw between subject random effects?

\item[\code{error}] see example

\item[\code{n}] number of simulations to perform
\end{ldescription}
\end{Arguments}
%
\begin{Details}\relax
This function generates the simulated dependent values for use in the VPC plotting function.
\end{Details}
%
\begin{Value}
a vector of simulated dependent variables (for us in the VPC plotting function)
\end{Value}
%
\begin{SeeAlso}\relax
\code{\LinkA{vpc}{vpc}}
\end{SeeAlso}
\inputencoding{utf8}
\HeaderA{theme\_empty}{Empty theme}{theme.Rul.empty}
%
\begin{Description}\relax
Empty theme
\end{Description}
%
\begin{Usage}
\begin{verbatim}
theme_empty()
\end{verbatim}
\end{Usage}
\inputencoding{utf8}
\HeaderA{theme\_plain}{Nicer default theme for ggplot2}{theme.Rul.plain}
%
\begin{Description}\relax
Nicer default theme for ggplot2
\end{Description}
%
\begin{Usage}
\begin{verbatim}
theme_plain()
\end{verbatim}
\end{Usage}
\inputencoding{utf8}
\HeaderA{triangle\_to\_full}{Lower to full triangle}{triangle.Rul.to.Rul.full}
%
\begin{Description}\relax
Convert the lower triangle of a covariance matrix to a full matrix object
\end{Description}
%
\begin{Usage}
\begin{verbatim}
triangle_to_full(vect)
\end{verbatim}
\end{Usage}
%
\begin{Arguments}
\begin{ldescription}
\item[\code{vect}] the lower triangle of a covariance matrix
\end{ldescription}
\end{Arguments}
\inputencoding{utf8}
\HeaderA{vpc}{VPC function}{vpc}
%
\begin{Description}\relax
Creates a VPC plot from observed and simulation data
\end{Description}
%
\begin{Usage}
\begin{verbatim}
vpc(sim = NULL, obs = NULL, psn_folder = NULL, bins = "jenks",
  n_bins = "auto", bin_mid = "mean", obs_cols = NULL, sim_cols = NULL,
  software = "auto", show = NULL, stratify = NULL, pred_corr = FALSE,
  pred_corr_lower_bnd = 0, pi = c(0.05, 0.95), ci = c(0.05, 0.95),
  uloq = NULL, lloq = NULL, log_y = FALSE, log_y_min = 0.001,
  xlab = NULL, ylab = NULL, title = NULL, facet_names = TRUE,
  smooth = TRUE, vpc_theme = NULL, facet = "wrap", labeller = NULL,
  vpcdb = FALSE, verbose = FALSE)
\end{verbatim}
\end{Usage}
%
\begin{Arguments}
\begin{ldescription}
\item[\code{sim}] a data.frame with observed data, containing the indenpendent and dependent variable, a column indicating the individual, and possibly covariates. E.g. load in from NONMEM using \LinkA{read\_table\_nm}{read.Rul.table.Rul.nm}

\item[\code{obs}] a data.frame with observed data, containing the indenpendent and dependent variable, a column indicating the individual, and possibly covariates. E.g. load in from NONMEM using \LinkA{read\_table\_nm}{read.Rul.table.Rul.nm}

\item[\code{psn\_folder}] instead of specyfing "sim" and "obs", specify a PsN-generated VPC-folder

\item[\code{bins}] either "density", "time", or "data", "none", or one of the approaches available in classInterval() such as "jenks" (default) or "pretty", or a numeric vector specifying the bin separators.

\item[\code{n\_bins}] when using the "auto" binning method, what number of bins to aim for

\item[\code{bin\_mid}] either "mean" for the mean of all timepoints (default) or "middle" to use the average of the bin boundaries.

\item[\code{obs\_cols}] observation dataset column names (list elements: "dv", "idv", "id", "pred")

\item[\code{sim\_cols}] simulation dataset column names (list elements: "dv", "idv", "id", "pred")

\item[\code{software}] name of software platform using (eg nonmem, phoenix)

\item[\code{show}] what to show in VPC (obs\_dv, obs\_ci, pi, pi\_as\_area, pi\_ci, obs\_median, sim\_median, sim\_median\_ci)

\item[\code{stratify}] character vector of stratification variables. Only 1 or 2 stratification variables can be supplied.

\item[\code{pred\_corr}] perform prediction-correction?

\item[\code{pred\_corr\_lower\_bnd}] lower bound for the prediction-correction

\item[\code{pi}] simulated prediction interval to plot. Default is c(0.05, 0.95),

\item[\code{ci}] confidence interval to plot. Default is (0.05, 0.95)

\item[\code{uloq}] Number or NULL indicating upper limit of quantification. Default is NULL.

\item[\code{lloq}] Number or NULL indicating lower limit of quantification. Default is NULL.

\item[\code{log\_y}] Boolean indicting whether y-axis should be shown as logarithmic. Default is FALSE.

\item[\code{log\_y\_min}] minimal value when using log\_y argument. Default is 1e-3.

\item[\code{xlab}] ylab as numeric vector of size 2

\item[\code{ylab}] ylab as numeric vector of size 2

\item[\code{title}] title

\item[\code{smooth}] "smooth" the VPC (connect bin midpoints) or show bins as rectangular boxes. Default is TRUE.

\item[\code{vpc\_theme}] theme to be used in VPC. Expects list of class vpc\_theme created with function vpc\_theme()

\item[\code{facet}] either "wrap", "columns", or "rows"

\item[\code{labeller}] ggplot2 labeller function to be passed to underlying ggplot object

\item[\code{vpcdb}] Boolean whether to return the underlying vpcdb rather than the plot

\item[\code{verbose}] show debugging information (TRUE or FALSE)
\end{ldescription}
\end{Arguments}
%
\begin{Value}
a list containing calculated VPC information (when vpcdb=TRUE), or a ggplot2 object (default)
\end{Value}
%
\begin{SeeAlso}\relax
\LinkA{sim\_data}{sim.Rul.data}, \LinkA{vpc\_cens}{vpc.Rul.cens}, \LinkA{vpc\_tte}{vpc.Rul.tte}
\end{SeeAlso}
\inputencoding{utf8}
\HeaderA{vpc\_cat}{VPC function for categorical}{vpc.Rul.cat}
%
\begin{Description}\relax
Creates a VPC plot from observed and simulation data
sim,
\end{Description}
%
\begin{Usage}
\begin{verbatim}
vpc_cat(sim = NULL, obs = NULL, psn_folder = NULL, bins = "jenks",
  n_bins = "auto", bin_mid = "mean", obs_cols = NULL, sim_cols = NULL,
  software = "auto", show = NULL, ci = c(0.05, 0.95), uloq = NULL,
  lloq = NULL, xlab = NULL, ylab = NULL, title = NULL, smooth = TRUE,
  stratify = NULL, stratify_color = NULL, vpc_theme = NULL,
  facet = "wrap", labeller = NULL, plot = TRUE, vpcdb = FALSE,
  verbose = FALSE)
\end{verbatim}
\end{Usage}
%
\begin{Arguments}
\begin{ldescription}
\item[\code{sim}] a data.frame with observed data, containing the indenpendent and dependent variable, a column indicating the individual, and possibly covariates. E.g. load in from NONMEM using \LinkA{read\_table\_nm}{read.Rul.table.Rul.nm}

\item[\code{obs}] a data.frame with observed data, containing the indenpendent and dependent variable, a column indicating the individual, and possibly covariates. E.g. load in from NONMEM using \LinkA{read\_table\_nm}{read.Rul.table.Rul.nm}

\item[\code{psn\_folder}] instead of specyfing "sim" and "obs", specify a PsN-generated VPC-folder

\item[\code{bins}] either "density", "time", or "data", "none", or one of the approaches available in classInterval() such as "jenks" (default) or "pretty", or a numeric vector specifying the bin separators.

\item[\code{n\_bins}] when using the "auto" binning method, what number of bins to aim for

\item[\code{bin\_mid}] either "mean" for the mean of all timepoints (default) or "middle" to use the average of the bin boundaries.

\item[\code{obs\_cols}] observation dataset column names (list elements: "dv", "idv", "id", "pred")

\item[\code{sim\_cols}] simulation dataset column names (list elements: "dv", "idv", "id", "pred")

\item[\code{software}] name of software platform using (eg nonmem, phoenix)

\item[\code{show}] what to show in VPC (obs\_ci, pi, pi\_as\_area, pi\_ci, obs\_median, sim\_median, sim\_median\_ci)

\item[\code{ci}] confidence interval to plot. Default is (0.05, 0.95)

\item[\code{uloq}] Number or NULL indicating upper limit of quantification. Default is NULL.

\item[\code{lloq}] Number or NULL indicating lower limit of quantification. Default is NULL.

\item[\code{xlab}] ylab as numeric vector of size 2

\item[\code{ylab}] ylab as numeric vector of size 2

\item[\code{title}] title

\item[\code{smooth}] "smooth" the VPC (connect bin midpoints) or show bins as rectangular boxes. Default is TRUE.

\item[\code{stratify}] character vector of stratification variables. Only 1 or 2 stratification variables can be supplied.

\item[\code{stratify\_color}] variable to stratify and color lines for observed data. Only 1 stratification variables can be supplied.

\item[\code{vpc\_theme}] theme to be used in VPC. Expects list of class vpc\_theme created with function vpc\_theme()

\item[\code{facet}] either "wrap", "columns", or "rows"

\item[\code{labeller}] ggplot2 labeller function to be passed to underlying ggplot object

\item[\code{plot}] Boolean indicting whether to plot the ggplot2 object after creation. Default is FALSE.

\item[\code{vpcdb}] boolean whether to return the underlying vpcdb rather than the plot

\item[\code{verbose}] show debugging information (TRUE or FALSE)
\end{ldescription}
\end{Arguments}
%
\begin{Value}
a list containing calculated VPC information (when vpcdb=TRUE), or a ggplot2 object (default)
\end{Value}
%
\begin{SeeAlso}\relax
\LinkA{vpc}{vpc}
\end{SeeAlso}
\inputencoding{utf8}
\HeaderA{vpc\_cens}{VPC function for left- or right-censored data (e.g. BLOQ data)}{vpc.Rul.cens}
%
\begin{Description}\relax
Creates a VPC plot from observed and simulation data
sim,
\end{Description}
%
\begin{Usage}
\begin{verbatim}
vpc_cens(sim = NULL, obs = NULL, psn_folder = NULL, bins = "jenks",
  n_bins = 8, bin_mid = "mean", obs_cols = NULL, sim_cols = NULL,
  software = "auto", show = NULL, stratify = NULL,
  stratify_color = NULL, ci = c(0.05, 0.95), uloq = NULL, lloq = NULL,
  plot = FALSE, xlab = NULL, ylab = NULL, title = NULL, smooth = TRUE,
  vpc_theme = NULL, facet = "wrap", labeller = NULL, vpcdb = FALSE,
  verbose = FALSE)
\end{verbatim}
\end{Usage}
%
\begin{Arguments}
\begin{ldescription}
\item[\code{sim}] a data.frame with observed data, containing the indenpendent and dependent variable, a column indicating the individual, and possibly covariates. E.g. load in from NONMEM using \LinkA{read\_table\_nm}{read.Rul.table.Rul.nm}

\item[\code{obs}] a data.frame with observed data, containing the indenpendent and dependent variable, a column indicating the individual, and possibly covariates. E.g. load in from NONMEM using \LinkA{read\_table\_nm}{read.Rul.table.Rul.nm}

\item[\code{psn\_folder}] instead of specyfing "sim" and "obs", specify a PsN-generated VPC-folder

\item[\code{bins}] either "density", "time", or "data", or a numeric vector specifying the bin separators.

\item[\code{n\_bins}] number of bins

\item[\code{bin\_mid}] either "mean" for the mean of all timepoints (default) or "middle" to use the average of the bin boundaries.

\item[\code{obs\_cols}] observation dataset column names (list elements: "dv", "idv", "id", "pred")

\item[\code{sim\_cols}] simulation dataset column names (list elements: "dv", "idv", "id", "pred")

\item[\code{software}] name of software platform using (eg nonmem, phoenix)

\item[\code{show}] what to show in VPC (obs\_ci, pi, pi\_as\_area, pi\_ci, obs\_median, sim\_median, sim\_median\_ci)

\item[\code{stratify}] character vector of stratification variables. Only 1 or 2 stratification variables can be supplied.

\item[\code{stratify\_color}] variable to stratify and color lines for observed data. Only 1 stratification variables can be supplied.

\item[\code{ci}] confidence interval to plot. Default is (0.05, 0.95)

\item[\code{uloq}] Number or NULL indicating upper limit of quantification. Default is NULL.

\item[\code{lloq}] Number or NULL indicating lower limit of quantification. Default is NULL.

\item[\code{plot}] Boolean indacting whether to plot the ggplot2 object after creation. Default is FALSE.

\item[\code{xlab}] ylab as numeric vector of size 2

\item[\code{ylab}] ylab as numeric vector of size 2

\item[\code{title}] title

\item[\code{smooth}] "smooth" the VPC (connect bin midpoints) or show bins as rectangular boxes. Default is TRUE.

\item[\code{vpc\_theme}] theme to be used in VPC. Expects list of class vpc\_theme created with function vpc\_theme()

\item[\code{facet}] either "wrap", "columns", or "rows"

\item[\code{labeller}] ggplot2 labeller function to be passed to underlying ggplot object

\item[\code{vpcdb}] boolean whether to return the underlying vpcdb rather than the plot

\item[\code{verbose}] show debugging information (TRUE or FALSE)
\end{ldescription}
\end{Arguments}
%
\begin{Value}
a list containing calculated VPC information, and a ggplot2 object
\end{Value}
%
\begin{SeeAlso}\relax
\LinkA{vpc}{vpc}
\end{SeeAlso}
\inputencoding{utf8}
\HeaderA{vpc\_tte}{VPC function for survival-type data}{vpc.Rul.tte}
%
\begin{Description}\relax
Creates a VPC plot from observed and simulation survival data
\end{Description}
%
\begin{Usage}
\begin{verbatim}
vpc_tte(sim = NULL, obs = NULL, psn_folder = NULL, rtte = FALSE,
  rtte_calc_diff = TRUE, events = NULL, bins = FALSE, n_bins = 10,
  software = "auto", obs_cols = NULL, sim_cols = NULL, kmmc = NULL,
  reverse_prob = FALSE, stratify = NULL, stratify_color = NULL,
  ci = c(0.05, 0.95), plot = FALSE, xlab = NULL, ylab = NULL,
  show = NULL, as_percentage = TRUE, title = NULL, smooth = FALSE,
  vpc_theme = NULL, facet = "wrap", labeller = NULL, verbose = FALSE,
  vpcdb = FALSE)
\end{verbatim}
\end{Usage}
%
\begin{Arguments}
\begin{ldescription}
\item[\code{sim}] a data.frame with observed data, containing the indenpendent and dependent variable, a column indicating the individual, and possibly covariates. E.g. load in from NONMEM using \LinkA{read\_table\_nm}{read.Rul.table.Rul.nm}

\item[\code{obs}] a data.frame with observed data, containing the indenpendent and dependent variable, a column indicating the individual, and possibly covariates. E.g. load in from NONMEM using \LinkA{read\_table\_nm}{read.Rul.table.Rul.nm}

\item[\code{psn\_folder}] instead of specyfing "sim" and "obs", specify a PsN-generated VPC-folder

\item[\code{rtte}] repeated time-to-event data? Deafult is FALSE (treat as single-event TTE)

\item[\code{rtte\_calc\_diff}] recalculate time (T/F)? When simulating in NONMEM, you will probably need to set this to TRUE to recalculate the TIME to relative times between events (unless you output the time difference between events and specify that as independent variable to the vpc\_tte() funciton.

\item[\code{events}] numeric vector describing which events to show a VPC for when repeated TTE data, e.g. c(1:4). Default is NULL, which shows all events.

\item[\code{bins}] either "density", "time", or "data", or a numeric vector specifying the bin separators.

\item[\code{n\_bins}] number of bins

\item[\code{software}] name of software platform using (eg nonmem, phoenix)

\item[\code{obs\_cols}] observation dataset column names (list elements: "dv", "idv", "id", "pred")

\item[\code{sim\_cols}] simulation dataset column names (list elements: "dv", "idv", "id", "pred")

\item[\code{kmmc}] either NULL (for regular TTE vpc, default), or a variable name for a KMMC plot (e.g. "WT")

\item[\code{reverse\_prob}] reverse the probability scale (i.e. plot 1-probability)

\item[\code{stratify}] character vector of stratification variables. Only 1 or 2 stratification variables can be supplied. If stratify\_color is also specified, only 1 additional stratification can be specified.

\item[\code{stratify\_color}] variable to stratify and color lines for observed data. Only 1 stratification variables can be supplied.

\item[\code{ci}] confidence interval to plot. Default is (0.05, 0.95)

\item[\code{plot}] Boolean indacting whether to plot the ggplot2 object after creation. Default is FALSE.

\item[\code{xlab}] ylab as numeric vector of size 2

\item[\code{ylab}] ylab as numeric vector of size 2

\item[\code{show}] what to show in VPC (obs\_ci, obs\_median, sim\_median, sim\_median\_ci)

\item[\code{as\_percentage}] Show y-scale from 0-100 percent? TRUE by default, if FALSE then scale from 0-1.

\item[\code{title}] title

\item[\code{smooth}] "smooth" the VPC (connect bin midpoints) or show bins as rectangular boxes. Default is TRUE.

\item[\code{vpc\_theme}] theme to be used in VPC. Expects list of class vpc\_theme created with function vpc\_theme()

\item[\code{facet}] either "wrap", "columns", or "rows"

\item[\code{labeller}] ggplot2 labeller function to be passed to underlying ggplot object

\item[\code{verbose}] TRUE or FALSE (default)

\item[\code{vpcdb}] Boolean whether to return the underlying vpcdb rather than the plot
\end{ldescription}
\end{Arguments}
%
\begin{Value}
a list containing calculated VPC information, and a ggplot2 object
\end{Value}
%
\begin{SeeAlso}\relax
\LinkA{vpc}{vpc}
\end{SeeAlso}
%
\begin{Examples}
\begin{ExampleCode}
## Example for repeated) time-to-event data
## with NONMEM-like data (e.g. simulated using a dense grid)
data(rtte_obs_nm)
data(rtte_sim_nm)

# treat RTTE as TTE, no stratification
vpc_tte(sim = rtte_sim_nm,
        obs = rtte_obs_nm,
        rtte = FALSE,
        sim_cols=list(dv = "dv", idv = "t"), obs_cols=list(idv = "t"))

# stratified for covariate and study arm
vpc_tte(sim = rtte_sim_nm,
        obs = rtte_obs_nm,
        stratify = c("sex","drug"),
        rtte = FALSE,
        sim_cols=list(dv = "dv", idv = "t"), obs_cols=list(idv = "t"))

# stratified per event number (we'll only look at first 3 events) and stratify per arm
vpc_tte(sim = rtte_sim_nm,
        obs = rtte_obs_nm,
        rtte = TRUE, events = c(1:3),
        stratify = c("drug"),
        sim_cols=list(dv = "dv", idv = "t"), obs_cols=list(idv = "t"))
\end{ExampleCode}
\end{Examples}
\printindex{}
\end{document}
